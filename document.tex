\documentclass[paper=a5,fontsize=11pt,parskip=half,twoside,DIV=16]{scrbook}

\KOMAoptions{toc=flat}
\KOMAoptions{toc=idx}

\usepackage{helpers}
\usepackage{version-and-copyright}
\usepackage{recipe}

\usepackage{hyperxmp}
\usepackage[type={CC},modifier={by-sa},version={4.0}]{doclicense}

\usepackage{makeidx}
\makeindex

\usepackage[toc,style=altlist,nopostdot,nomain]{glossaries}
\newglossary*{term}{Cooking Terms}
\newglossary*{tech}{Cooking Techniques}
\newglossary*{eqpt}{Equipment}
\makeglossaries
\loadglsentries[term]{term-glossary}
\loadglsentries[tech]{tech-glossary}
\loadglsentries[eqpt]{eqpt-glossary}
\glsaddall

\title{The Open Source Cookbook}

\lowertitleback{This compilation \copyrightRange{2020} Samuel Flint,
  licensed \doclicenseName (\doclicenseIcon).  Recipes copyright their
  respective contributors, and under the same terms unless otherwise
  noted.

  Revision \texttt{\gitrevision}.
}

\usepackage{hyperref}
\hypersetup{
  pdftitle={The Open Source Cookbook},
  pdfcreator={LatexMk},
  pdflang={en-US},
  colorlinks=true,
  linkcolor=black,
  citecolor=black,
  filecolor=black,
  urlcolor=black
}

\begin{document}

\maketitle

\frontmatter
\tableofcontents

\mainmatter

\chapter{Basics}
\label{cha:basics}

%%% Local Variables:
%%% mode: latex
%%% TeX-master: "document"
%%% End:

\chapter{Salads \& Dressings}
\label{cha:salads--dressings}

%%% Local Variables:
%%% mode: latex
%%% TeX-master: "document"
%%% End:

\chapter{Soups}
\label{cha:soups}

%%% Local Variables:
%%% mode: latex
%%% TeX-master: "document"
%%% End:

\chapter{Appetizers \& Hors d'\oe{}uvres}
\label{cha:appetizers}

%%% Local Variables:
%%% mode: latex
%%% TeX-master: "document"
%%% End:

\chapter{Main Dishes \& Entrees}
\label{cha:mains-entrees}

%%% Local Variables:
%%% mode: latex
%%% TeX-master: "document"
%%% End:

\chapter{Side Dishes}
\label{cha:side-dishes}

%%% Local Variables:
%%% mode: latex
%%% TeX-master: "document"
%%% End:

\chapter{Desserts}
\label{cha:desserts}

%%% Local Variables:
%%% mode: latex
%%% TeX-master: "document"
%%% End:

\chapter{Snacks, \&c.}
\label{cha:snacks-etc}

%%% Local Variables:
%%% mode: latex
%%% TeX-master: "document"
%%% End:


\backmatter

\chapter{Cooking for Groups}
\label{cha:cooking-groups}

%%% Local Variables:
%%% mode: latex
%%% TeX-master: "document"
%%% End:

\chapter{Suggested Menus}
\label{cha:suggested-menus}

%%% Local Variables:
%%% mode: latex
%%% TeX-master: "document"
%%% End:


% Glossaries
\printglossary[type=tech,nonumberlist]
\printglossary[type=eqpt,nonumberlist]
\printglossary[type=term,nonumberlist]

% Index
\printindex

\end{document}


%%% Local Variables:
%%% mode: latex
%%% TeX-master: t
%%% End:
