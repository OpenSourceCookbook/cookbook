\documentclass[paper=a5,fontsize=11pt,parskip=half,twoside,DIV=16]{scrbook}

\KOMAoptions{toc=flat}
\KOMAoptions{toc=idx}

\usepackage{helpers}
\usepackage{version-and-copyright}
\usepackage{recipe}

\usepackage{hyperxmp}
\usepackage[type={CC},modifier={by-sa},version={4.0}]{doclicense}

\usepackage{makeidx}
\makeindex

\usepackage[toc,style=altlist,nopostdot,nomain]{glossaries}
\newglossary*{term}{Cooking Terms}
\newglossary*{tech}{Cooking Techniques}
\newglossary*{eqpt}{Equipment}
\makeglossaries
\loadglsentries[term]{term-glossary}
\loadglsentries[tech]{tech-glossary}
\loadglsentries[eqpt]{eqpt-glossary}
\glsaddall

\title{The Open Source Cookbook}

\lowertitleback{This compilation \copyrightRange{2020} Samuel Flint
  and the named contributors.

  This compilation is licensed under
  \doclicenseLongNameRef\ (\doclicenseIcon).

  You should have received a copy of the license along with this
  work.  If not, see \href{\doclicenseURL}{\texttt{\doclicenseURL}}.

  Revision \texttt{\gitrevision}.
}

\usepackage{hyperref}
\hypersetup{
  pdftitle={The Open Source Cookbook},
  pdfcreator={LatexMk},
  pdflang={en-US},
  colorlinks=true,
  linkcolor=black,
  citecolor=black,
  filecolor=black,
  urlcolor=black
}

\begin{document}

\maketitle

\frontmatter
\tableofcontents

\mainmatter

\chapter{Basics}
\label{cha:basics}

%%% Local Variables:
%%% mode: latex
%%% TeX-master: "document"
%%% End:

\chapter{Breads}
\label{cha:breads}

%%% Local Variables:
%%% mode: latex
%%% TeX-master: "document"
%%% End:

\chapter{Salads \& Dressings}
\label{cha:salads--dressings}

%%% Local Variables:
%%% mode: latex
%%% TeX-master: "document"
%%% End:

\chapter{Soups}
\label{cha:soups}

%%% Local Variables:
%%% mode: latex
%%% TeX-master: "document"
%%% End:

\chapter{Appetizers \& Hors d'\oe{}uvres}
\label{cha:appetizers}

%%% Local Variables:
%%% mode: latex
%%% TeX-master: "document"
%%% End:

\chapter{Main Dishes \& Entrees}
\label{cha:mains-entrees}

\begin{recipe}{Beef Stroganoff}{}{5--6}
  \index{beef}
  \begin{ingredients}
    \ingredient{stew beef, \half-inch cubes}{1-\half~lb}
    \ingredient{butter}{2 tbsp}
    \ingredient{large white onion, thinly sliced}{1}
    \ingredient{mushrooms, button or portobella}{1 lb}
    \ingredient{minced garlic}{1 tbsp}
    \ingredient{sour cream, full fat}{1 qt}
    \ingredient{soy sauce}{\quarter cup, divided}
    \ingredient{Worcestershire sauce}{\quarter cup, divided}
    \ingredient{fresh grated nutmeg}{2 tsp}
    \ingredient{kosher salt}{to taste}
    \ingredient{black pepper}{to taste}
  \end{ingredients}
  \begin{enumerate}
  \item In a large, heavy-bottomed skillet on medium heat, melt
    butter.
  \item To melted butter, add onion, garlic and tbsp each of soy \&
    Worcestershire sauces.  Saut\'ee until onion becomes translucent
    and some slices begin to brown.
  \item While saut\'eeing onion, season meat liberally with salt,
    pepper and a tbsp each of soy \& Worcestershire sauce, ensure
    evenly covered.
  \item Remove onions to plate.  Replace with mushrooms, and season
    with salt, pepper and a tbsp each of soy \& Worcestershire sauce.
    Allow to brown and begin to soften.  Remove to plate with onions
  \item If pan seems lacking in fat, add no more than 1 tbsp of
    additional butter and raise heat to medium-high.
  \item Add seasoned meat, allow to brown well on all sides.
  \item In a medium-sized bowl, add remaining ingredients and stir
    well to form sauce.  Do not try to thin out additionally.
    Remember to season to taste.
  \item Reduce heat to medium-low.  Add onions, mushrooms and sour
    cream sauce, mixing well.
  \item Keep at a low simmer until meat is cooked through and tender.
  \item If sauce is over thin, thicken with a corn-starch slurry.
  \end{enumerate}
  Serve over white rice or buttered egg noodles.
  \begin{note}
    \index{vegetarian}
    Stroganoff may be made vegetarian by replacing beef with an
    additional 1-\half~lb mushrooms, though be sure to use several
    varieties.
  \end{note}
\end{recipe}

%%% Local Variables:
%%% mode: latex
%%% TeX-master: "document"
%%% End:

\chapter{Side Dishes}
\label{cha:side-dishes}

%%% Local Variables:
%%% mode: latex
%%% TeX-master: "document"
%%% End:

\chapter{Desserts}
\label{cha:desserts}

%%% Local Variables:
%%% mode: latex
%%% TeX-master: "document"
%%% End:

\chapter{Snacks, \&c.}
\label{cha:snacks-etc}

%%% Local Variables:
%%% mode: latex
%%% TeX-master: "document"
%%% End:

% Maybe a "Punches chapter?"

\backmatter

\chapter{Cooking for Groups}
\label{cha:cooking-groups}

%%% Local Variables:
%%% mode: latex
%%% TeX-master: "document"
%%% End:

\chapter{Suggested Menus}
\label{cha:suggested-menus}

%%% Local Variables:
%%% mode: latex
%%% TeX-master: "document"
%%% End:


% Glossaries (consider re-ordering)
\printglossary[type=tech,nonumberlist] %Techniques
\printglossary[type=eqpt,nonumberlist] %Equiptment
\printglossary[type=term,nonumberlist] %Terms

% Index
\printindex

\end{document}


%%% Local Variables:
%%% mode: latex
%%% TeX-master: t
%%% End:
