\chapter{Soups}
\label{cha:soups}

\begin{recipe}{French Onion Soup}{}{12}
  \index{French}
  \index{traditional}
  \index{nut-free}
  \index{wheat-free}
  \begin{ingredients}
    % Prep the base
    \ingredient{butter}{2 tbsp}
    \ingredient{onion, diced, fine}{1}
    \ingredient{garlic, minced}{1 tbsp}
    \ingredient{tomato paste}{1 tbsp}
    \ingredient{fresh thyme}{5 sprigs}
    \ingredient{brandy}{2 fl oz}
    \ingredient{bay leaf}{1}
    \ingredient{beef soup bones}{1 lb}
    \ingredient{beef stock}{3 qt}
    % Make the soup
    \ingredient{onion, sliced thin}{3 lb}
    % Garnish
    \ingredient{baguette}{1 loaf}
    \ingredient{butter, softened}{1 stick}
    \ingredient{Gruy\`ere cheese, shredded}{to taste}
  \end{ingredients}
  \majorstep{Prepare the Soup}
  \begin{enumerate}
  \item Melt 2 tbsp butter in 5 quart Dutch oven on medium heat.
  \item Gently cook onion until translucent.  Add garlic and tomato
    paste, mix well and allow tomato paste to darken slightly.
  \item Add brandy, allowing alcohol to cook out.  Stir well to loosen
    anything stuck to the bottom.
  \item Add in bay leaf, thyme, soup bones and stock.  Bring to a
    rolling boil for 10 minutes and reduce to a simmer for at least 2
    hours.
  \item 30 minutes before adding onions to stock, soak in well-salted
    water.
  \item After onions have soaked, remove soup bones from stock and add
    well-drained onions.
  \item Bring to rapid boil for 10 minutes, reduce heat to simmer and
    allow to simmer for 2-3 hours, or until onions are
    well-caramelized.
  \item Serve soup piping hot and garnished with homemade croutons and
    shredded Gruy\`ere cheese.
  \end{enumerate}
  \majorstep{Prepare Garnishes}
  \begin{enumerate}
  \item Preheat oven to 400F.
  \item Cut baguette into \half-inch slices.
  \item Butter each side of slices, seasoning with salt, pepper and
    garlic powder to taste.
  \item Arrange on parchment-lined baking sheet, bake in oven 25
    minutes, or until golden brown.
  \item Remove and allow to cool slightly.
  \end{enumerate}
  \begin{note}
    Garnish with seasoned croutons (made as above) and Gruy\`ere by
    ladelling into bowls, floating crouton on top and sprinkling
    heavily with cheese.  Allow cheese to melt under broiler, and
    serve immediately.
  \end{note}
\end{recipe}

%%% Local Variables:
%%% mode: latex
%%% TeX-master: "document"
%%% End:
